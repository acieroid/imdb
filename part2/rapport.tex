\documentclass[a4paper,12pt]{article}
\usepackage[utf8]{inputenc}
\usepackage[T1]{fontenc} 
\usepackage[french]{babel}
\usepackage{graphicx}
\usepackage{amsmath}
\usepackage{a4wide}
\usepackage{subfig}
\usepackage{wrapfig}
\usepackage{url}
\usepackage[pdftex]{hyperref}
\hypersetup{
  colorlinks,
  citecolor=black,
  filecolor=black,
  linkcolor=black,
  urlcolor=black
}

\renewcommand{\baselinestretch}{1.05}

\title{INFO-H-303: projet IMDB -- première partie}

\author{Quentin \textsc{Stiévenart} \and Damien \textsc{Wiltgen}}
\date{\today}

\newcommand{\HRule}{\rule{\linewidth}{0.5mm}}

\begin{document}

\maketitle

\HRule

\section{Modèle entité-association}
% TODO: refaire le schéma qu'ils ont rendu avec dia/inkscape
\subsection{Contraintes d'intégrité}
\section{Modèle relationnel}
% TODO: simplement une réécriture du create.sql
\section{Méthode d'extraction des données}
Pour extraire les données depuis les fichiers d'\emph{IMDB}, Perl a
été choisi, grâce à la faciliter d'écrire des programmes manipulant du
texte ainsi qu'à la présence du module \texttt{DBI}, qui permet de
manipuler plusieurs moteurs de base de donnée en passant par une
interface unique. Il a ainsi pu être possible de tester la vitesse
d'importation de données sur plusieurs moteurs de base de données, et
le choix de moteur s'est posé sur \emph{SQLite} pour plusieurs
raisons:

\begin{list}{-}{}
  \item Les tests de contraintes étrangères sont désactivés par
    défaut, ce qui a pour effet de grandement améliorer le temps
    d'import des données. Il est possible de désactiver ces tests dans
    d'autres moteurs, mais cela dépend des moteurs et varie en
    complexité.
  \item Comme la base de donnée est simplement stockée dans un
    fichier, il est très simple de la mettre en place sur une nouvelle
    machine, d'en faire des sauvegardes, et de repartir avec une base
    de données vide.
  \item L'API \emph{SQLite} de Python (le langage utilisé pour
    l'interface web) est très simple et est fournie avec Python.
\end{list}

Lors de l'import, la base de donnée a été placée en RAM (dans
\texttt{/dev/shm/}) afin de ne pas être limité par la vitesse d'écriture du
disque dur.

Les valeurs pouvant être nulles dans le diagramme entité-associations
sont remplacées par des chaînes de caractère vides dans le cas où sont
valeurs sont des chaînes de caractères, ou par la valeur \texttt{null}
dans le cas de nombres.
\section{Requêtes demandées}
\section{Hypothèses}

\end{document}
