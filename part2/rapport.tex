\documentclass[a4paper,12pt]{article}
\usepackage[utf8]{inputenc}
\usepackage[T1]{fontenc} 
\usepackage[french]{babel}
\usepackage{graphicx}
\usepackage{amsmath}
\usepackage{a4wide}
\usepackage{subfig}
\usepackage{wrapfig}
\usepackage{url}
\usepackage{latexsym} % \Join
\usepackage[pdftex]{hyperref}
\hypersetup{
  colorlinks,
  citecolor=black,
  filecolor=black,
  linkcolor=black,
  urlcolor=black
}
\usepackage[top=2.5cm, bottom=2.5cm, left=2.5cm, right=2.5cm]{geometry}
%\renewcommand{\baselinestretch}{1.05}

\usepackage{listings}

\lstset{ %
language=SQL,
showspaces=false
showstringspaces=false,
showtabs=false,
tabsize=2,
}


\title{INFO-H-303: projet IMDB -- première partie}

\author{Quentin \textsc{Stiévenart} \and Damien \textsc{Wiltgen}}
\date{\today}

\newcommand{\HRule}{\rule{\linewidth}{0.5mm}}

\begin{document}

\maketitle

\HRule

\section{Modèle entité-association}
% TODO: refaire le schéma qu'ils ont rendu avec dia/inkscape
\subsection{Contraintes d'intégrité}
\section{Modèle relationnel}
% TODO: simplement une réécriture du create.sql
\section{Méthode d'extraction des données}
Pour extraire les données depuis les fichiers d'\emph{IMDB}, Perl a
été choisi, grâce à la faciliter d'écrire des programmes manipulant du
texte ainsi qu'à la présence du module \texttt{DBI}, qui permet de
manipuler plusieurs moteurs de base de donnée en passant par une
interface unique. Il a ainsi pu être possible de tester la vitesse
d'importation de données sur plusieurs moteurs de base de données, et
le choix de moteur s'est posé sur \emph{SQLite} pour plusieurs
raisons:

\begin{list}{-}{}
  \item Les tests de contraintes étrangères sont désactivés par
    défaut, ce qui a pour effet de grandement améliorer le temps
    d'import des données. Il est possible de désactiver ces tests dans
    d'autres moteurs, mais cela dépend des moteurs et varie en
    complexité.
  \item Comme la base de donnée est simplement stockée dans un
    fichier, il est très simple de la mettre en place sur une nouvelle
    machine, d'en faire des sauvegardes, et de repartir avec une base
    de données vide.
  \item L'API \emph{SQLite} de Python (le langage utilisé pour
    l'interface web) est très simple et est fournie avec Python.
\end{list}

Lors de l'import, la base de donnée a été placée en RAM (dans
\texttt{/dev/shm/}) afin de ne pas être limité par la vitesse d'écriture du
disque dur.

Les valeurs pouvant être nulles dans le diagramme entité-associations
sont remplacées par des chaînes de caractère vides dans le cas où sont
valeurs sont des chaînes de caractères, ou par la valeur \texttt{null}
dans le cas de nombres.
\section{Requêtes demandées}
Les requêtes demandées étaient les suivantes:
\begin{list}{-}{}
  \item R1 : Les acteurs qui ont joué toutes les années entre 2003 et
    2007.
  \item R2 : Les auteurs qui ont écrit au moins deux films pendant la
    même année.
  \item R3 : Les acteurs Y qui sont à une distance 2 d'un acteur X. Un
    acteur Y est à une distance 1 d'un acteur X si ces deux acteurs
    ont joué dans le même film.
  \item R4 : Les épisodes de série où il n'y a aucun acteur masculin.
  \item R5 : Les acteurs qui ont joué dans le plus de séries.
  \item R6 : Les séries avec leur nombre total d'épisodes, le nombre
    d'épisodes moyen par an et le nombre d'acteurs moyen par saison
    depuis leur année de création et ce pour toutes les séries dont la
    note est supérieure à la moyenne des notes des séries.
\end{list}
\subsection{Algèbre relationnelle}
\begin{list}{-}{}
  \item R1 :
    \begin{align*}
      WorkInPeriod & \leftarrow \sigma_{Year \geq 2003 \wedge Year \leq 2007} (Work) \\
      AllYears & \leftarrow \pi_{Year} (WorkInPeriod)\\
      R1 & \leftarrow \pi_{FirstName, LastName, Num, Year} (Actor * WorkInPeriod) / AllYears
    \end{align*}
  \item R2 :
    \begin{align*}
      MovieWriter & \leftarrow Writer * (Work * Movie) \\
      MovieWriter' & \leftarrow \alpha_{ID,Title:ID2,Title2} (MovieWriter) \\
      R2 & \leftarrow \pi_{FirstName,LastName,Num} (MovieWriter \Join_{ID \neq ID2 \wedge Title \neq Title2} MovieWriter')
    \end{align*}
  \item R3 :
    Si les infos sur l'acteur X sont dans $XFName$, $XLName$, $XNum$
    \begin{align*}
      X & \leftarrow \sigma_{FirstName = XFName \wedge LastName = XLName \wedge Num = XNum} (Actor) \\
      R3 & \leftarrow \pi_{FirstName,LastName,Num}(Actor*\pi_{ID} (X * (Work * Movie))
    \end{align*}
  \item R4 :
    \begin{align*}
      R4 & \leftarrow \pi_{ID} (Episodes) - \pi_{ID} ((\sigma_{Gender='M'} Person) * Actor * Episode)
    \end{align*}
\end{list}
\subsection{Calcul relationnel tuple}
\subsection{SQL}
\begin{list}{-}{}
  \item R1 :
    \begin{lstlisting}
select FirstName, LastName, Num from Actor where
  count(select distinct(Year) from Work
          where Year >= 2003 
            and Year <= 2007
            and Actor.ID = Work.ID) = (2007-2003)

    \end{lstlisting}
  \item R2 :
% lolwat :(
    \begin{lstlisting}
select W1.FirstName, W1.LastName, W1.Num
  from Writer W1, Writer W2
  where W1.FirstName = W2.FirstName 
    and W1.LastName = W2.LastName
    and W1.Num = W2.Num
    and count(select * from Work Work1, Work Work2, Movie M1, MovieM2 
                where W1.ID = M1.ID
                  and W2.ID = M2.ID
                  and W1.ID != W2.ID
                  and Work1.ID = M1.ID
                  and Work2.ID = M2.ID
                  and Work1.Year >= 2003
                  and Work2.Year <= 2007) >= 2
    \end{lstlisting}
  \item R3 :
    \begin{lstlisting}
select X.FirstName, X.LastName, X.Num from Actor X
  where exists (select * from Actor Y
                  where exists (select * from Actor Z, 
                                  where Z.ID = Y.ID)
                    and Y.ID = X.ID)
    \end{lstlisting}
  \item R4 :
    \begin{lstlisting}
select ID from Episode
  where not exists (select * from Person, Actor
                      where Person.FirstName = Actor.FirstName
                        and Person.LastName = Actor.LastName
                        and Person.Num = Actor.Num
                        and Actor.ID = Episode.ID)
    \end{lstlisting}

\section{Hypothèses}
\end{list}

\end{document}
